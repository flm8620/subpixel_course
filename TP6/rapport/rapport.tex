\documentclass{article}
\usepackage[utf8]{inputenc}
\usepackage{graphicx}
\usepackage{amsmath}
\usepackage{amsfonts}
\usepackage{amsbsy}
\usepackage{mathrsfs}
\usepackage{appendix}
\usepackage{amsthm}
\usepackage{bbold}
\usepackage{epstopdf}
\usepackage{stmaryrd}

\newcommand{\bs}[1]{\boldsymbol{#1}}
\newcommand{\bb}[1]{\mathbf{#1}}
\newcommand{\pd}[2]{\frac{\partial {#1}}{\partial {#2}}}
\newcommand{\parti}[1]{\pd{}{#1}}
\newcommand{\bigCI}{\mathrel{\text{\scalebox{1.07}{$\perp\mkern-10mu\perp$}}}}
\title{TP6 for Sub-pixel}
\author{Leman FENG\\ Email: flm8620@gmail.com\\Website: lemanfeng.com}

\begin{document}
	\maketitle
	\section*{Ex 15}
	\subsection*{1}
	Notice that $\frac{d x_+^{n+1}}{d x} = (n+1)x_+^{n}, x\neq 0$. So inversely,
	$$
	\int_a^b x_+^n = \frac{1}{n+1} [x^{n+1}_+]_a^b
	$$
	\subsection*{2}
	It's easy to verify this is true for $n=1$.
	Assume it's true for $n$.
	\begin{equation*}
	\begin{split}
	\beta^{n+1}(y) &= \beta^n \star \beta^0 = \frac{1}{n!}\sum_{k=0}^{n+1} \binom{n+1}{k} (-1)^k \int (x-k+\frac{n+1}{2})_+^n \cdot \mathbb{1}_{[-1/2,1/2[}(y-x) dx\\
	&=\frac{1}{n!}\sum_{k=0}^{n+1} \binom{n+1}{k} (-1)^k \int_{y-1/2}^{y+1/2} (x-k+\frac{n+1}{2})_+^n dx\\
	&=\frac{1}{n!}\sum_{k=0}^{n+1} \binom{n+1}{k} (-1)^k \frac{1}{n+1} \int_{y-k+n/2}^{y-k+n/2+1} x_+^{n+1} dx\\
	&=\frac{1}{(n+1)!}\sum_{k=0}^{n+1} \binom{n+1}{k} (-1)^k [x_+^{n+1}]_{y-k+n/2}^{y-k+n/2+1}\\
	&=\frac{1}{(n+1)!}\sum_{k=0}^{n+1} \binom{n+1}{k} (-1)^k ((y-k+n/2+1)_+^{n+1} - (y-k+n/2)_+^{n+1})\\
	&=\frac{1}{(n+1)!}(\sum_{k=0}^{n+1} \binom{n+1}{k} (-1)^k (y-k+(n+2)/2)_+^{n+1} +  \sum_{k=0}^{n+1} \binom{n+1}{k} (-1)^{k+1} (y-(k+1)+(n+2)/2)_+^{n+1})\\
	&=\frac{1}{(n+1)!}(\sum_{k=0}^{n+1} \binom{n+1}{k} (-1)^k (y-k+(n+2)/2)_+^{n+1} +  \sum_{k=1}^{n+2} \binom{n+1}{k-1} (-1)^{k} (y-k+(n+2)/2)_+^{n+1})\\
	\end{split}
	\end{equation*}
	
	We know that
	\begin{equation*}
	\begin{cases}
	\binom{n+2}{k} = \binom{n+1}{k} + \binom{n+1}{k-1} & 1\leq k \leq n+1\\
	\binom{n+2}{k} = \binom{n+1}{k-1} & \leq k = n+2\\
	\binom{n+2}{k} = \binom{n+1}{k} & \leq k = 0
	\end{cases}
	\end{equation*}
	
	So we have
	\begin{equation*}
	\begin{split}
	\beta^{n+1}(y)
	&=\frac{1}{(n+1)!}\sum_{k=0}^{n+2} \binom{n+2}{k} (-1)^k (y-k+(n+2)/2)_+^{n+1}\\
	\end{split}
	\end{equation*}
	so this is also true for $n+1$.
	By recurrence, this is true for all $n\geq 1$
	\subsection*{3}
	\begin{equation*}
	\begin{split}
	\forall p\in \mathbb{Z},\ \hat{u}[p]&=\sum_{l=0}^{N-1} u[l] e^{-i\frac{2\pi l}{N}p}\\
	&=\sum_{l=0}^{N-1} e^{-i\frac{2\pi l}{N}p} \sum_{k\in\mathbb{Z}} v[k]w[l-k] \\
	&=\sum_{l=0}^{N-1} e^{-i\frac{2\pi l}{N}p} \sum_{k\in\mathbb{Z}} v[l-k]w[k] \\
	&=\sum_{k\in\mathbb{Z}} w[k]  e^{-i\frac{2\pi k}{N}p} \sum_{l=0}^{N-1}  v[l-k] e^{-i\frac{2\pi (l-k)}{N}p}  \\
	&=\sum_{k\in\mathbb{Z}} w[k]  e^{-i\frac{2\pi k}{N}p} \sum_{l=0}^{N-1}  v[(l-k)\mod N] e^{-i\frac{2\pi (l-k)\mod N}{N}p}  \\
	&=\sum_{k\in\mathbb{Z}} w[k]  e^{-i\frac{2\pi k}{N}p} \hat{v}[p]\\
	&=\hat{w}(\frac{2\pi k}{N})\hat{v}[p]\\
	\end{split}
	\end{equation*}
	
	\subsection*{4}
	Assume there exist a $c$ such that $U(x)$ interpolate exactly $u$
	
	\begin{equation*}
		\begin{split}
		\forall l\in \mathbb{Z},\ u[l]&=U(l)
		= \sum_{k\in\mathbb{Z}} c[k] \beta^n(l-k) = \sum_{k\in\mathbb{Z}} c[k] \beta^n_d[l-k]\\
		\end{split}
	\end{equation*}
	use the result from 3:
	\begin{equation*}
	\begin{split}
	\forall p\in \mathbb{Z},\ \hat{u}[p]&= \hat{c}[p] \hat{\beta^n_d}(\frac{2\pi p}{N})\\
	\hat{c}[p] &= \frac{\hat{u}[p]}{\hat{\beta^n_d}(\frac{2\pi p}{N})}
	\end{split}
	\end{equation*}
	where $\hat{\beta^n_d}$ is the $2\pi$-periodic of $\hat{\beta^n}$.
	
	
	So we proved that there is one way to construct such $c$, and for all $c$ that satisfy this property, its fourier transform must be in the above form, so $c$ exist and is unique.
	
	\subsection*{5}
	\begin{equation*}
	\begin{split}
	\forall 0\leq l\leq 2N-1,\ u_2[l]&=U(l/2)
	= \sum_{k\in\mathbb{Z}} c[k] \beta^n(l/2-k)\\
	\end{split}
	\end{equation*}
	
	\begin{equation*}
	\begin{split}
	\forall 0\leq p\leq 2N-1,\ \hat{u_2}[p]&=\sum_{l=0}^{2N-1} u_2[l] e^{-i\frac{2\pi l}{2N}p}\\
	&=\sum_{l=0}^{2N-1} \sum_{k\in\mathbb{Z}} c[k] \beta^n(l/2-k) e^{-i\frac{2\pi l}{2N}p}\\
	&=\sum_{a=0}^{N-1} \sum_{k\in\mathbb{Z}} c[k] \beta^n(a-k) e^{-i\frac{2\pi a}{N}p}
	+ \sum_{b=0}^{N-1} \sum_{k\in\mathbb{Z}} c[k] \beta^n(b-k+1/2) e^{-i\frac{2\pi b}{N}p} e^{-i\frac{\pi}{N}p}\\
	\end{split}
	\end{equation*}
	
	According to question 4, first term is simplified:
	\begin{equation*}
	\begin{split}
	\forall 0\leq p\leq 2N-1,\ \hat{u_2}[p]&=\hat{u}[p]
	+ \sum_{b=0}^{N-1} \sum_{k\in\mathbb{Z}} c[k] \beta^n(b-k+1/2) e^{-i\frac{2\pi b}{N}p} e^{-i\frac{\pi}{N}p}\\
	\end{split}
	\end{equation*}
	Note $\beta^n_T (x) = \beta^n(x-T)$
	
	\begin{equation*}
	\begin{split}
	\forall 0\leq p\leq 2N-1,\ \hat{u_2}[p]&=\hat{u}[p]
	+ e^{-i\frac{\pi}{N}p} \sum_{b=0}^{N-1} e^{-i\frac{2\pi b}{N}p}\sum_{k\in\mathbb{Z}} c[k]  \beta^n_{-\frac{1}{2}}(b-k)  \\
	&=\hat{u}[p]
	+ e^{-i\frac{\pi}{N}p} \sum_{b=0}^{N-1} e^{-i\frac{2\pi b}{N}p}\sum_{k\in\mathbb{Z}} c[b-k]  \beta^n_{-\frac{1}{2}}(k)  \\
	&=\hat{u}[p] + e^{-i\frac{\pi}{N}p} \sum_{k\in\mathbb{Z}} \beta^n_{-\frac{1}{2}}(k) e^{-i\frac{2\pi k}{N}p} \sum_{b=0}^{N-1} c[b-k] e^{-i\frac{2\pi (b-k)}{N}p}    \\
	&=\hat{u}[p] + e^{-i\frac{\pi}{N}p} \widehat{{\beta^n_{-\frac{1}{2}}}_d}(\frac{2\pi p}{N}) \hat{c}[p]  \\
	\end{split}
	\end{equation*}
	
	By properties of Fourier transform, we have $\hat{\beta^n_T}(\xi) =-\hat{\beta^n}(\xi)e^{i\xi T}$.
	
	$\widehat{{\beta^n_T}_d}$ is $2\pi$-periodic of $\hat{\beta^n_T}(\xi)$
	\begin{equation*}
	\begin{split}
	\widehat{{\beta^n_{-\frac{1}{2}}}_d}(\xi) &=\sum_{a\in\mathbb{Z}} \hat{\beta^n_{-\frac{1}{2}}}(\xi+2\pi a) = e^{i\xi \frac{1}{2}} \sum_{a\in\mathbb{Z}}  \hat{\beta^n}(\xi+2\pi a)  e^{i \pi a}= e^{i\xi \frac{1}{2}} \sum_{a\in\mathbb{Z}}  \hat{\beta^n}(\xi+2\pi a)  (-1)^a\\
	&=e^{i\xi \frac{1}{2}} (\sum_{a\in\mathbb{Z}}  \hat{\beta^n}(\xi+2\pi a) - 2 \sum_{b\in\mathbb{Z}}  \hat{\beta^n}(\xi+4\pi b+2\pi))\\
	&=e^{i\xi \frac{1}{2}} (\hat{\beta^n_d}(\xi) - 2 \hat{\beta^n_{\frac{d}{2}}}(\xi + 2\pi))\\
	\end{split}
	\end{equation*}
	where $\hat{\beta^n_{\frac{d}{2}}}$ is $4\pi$-periodic of $\hat{\beta^n}$.
	
	So
	\begin{equation*}
	\begin{split}
	\widehat{{\beta^n_{-\frac{1}{2}}}_d}(\frac{2\pi p}{N}) &=e^{i\frac{\pi p}{N}} (\hat{\beta^n_d}(\frac{2\pi p}{N}) - 2 \hat{\beta^n_{\frac{d}{2}}}(\frac{2\pi p}{N} + 2\pi))\\
	\end{split}
	\end{equation*}
	
	Then
	\begin{equation*}
	\begin{split}
	\forall 0\leq p\leq 2N-1,\ \hat{u_2}[p]&=\hat{u}[p] + \hat{c}[p] (\hat{\beta^n_d}(\frac{2\pi p}{N}) - 2 \hat{\beta^n_{\frac{d}{2}}}(\frac{2\pi p}{N} + 2\pi))  \\
	&=\hat{u}[p] + \frac{\hat{u}[p]}{\hat{\beta^n_d}(\frac{2\pi p}{N})}  (\hat{\beta^n_d}(\frac{2\pi p}{N}) - 2 \hat{\beta^n_{\frac{d}{2}}}(\frac{2\pi p}{N} + 2\pi))  \\
	&= \hat{u}[p] (2- 2 \hat{\beta^n_{\frac{d}{2}}}(\frac{2\pi p}{N} + 2\pi))
	\end{split}
	\end{equation*}
	
	We know that $\hat{\beta^n}(\xi) = (\frac{\sin (\xi/2)}{\xi/2})^{n+1}$
	Then
	
	\begin{equation*}
	\hat{\beta^n_{\frac{d}{2}}}(\xi) = \sum_{b\in\mathbb{Z}}  \hat{\beta^n}(\xi+4\pi b+2\pi) = \sum_{b\in\mathbb{Z}}  (\frac{\sin (\xi/2 +2\pi b +\pi)}{\xi/2 +2\pi b +\pi})^{n+1}= \sum_{b\in\mathbb{Z}}  (\frac{-\sin (\xi/2)}{\xi/2 +2\pi b +\pi})^{n+1}
	\end{equation*}
\end{document}
