\documentclass{article}
\usepackage[utf8]{inputenc}
\usepackage{graphicx}
\usepackage{amsmath}
\usepackage{amsfonts}
\usepackage{amsbsy}
\usepackage{mathrsfs}
\usepackage{appendix}
\usepackage{amsthm}
\usepackage{bbold}
\usepackage{epstopdf}
\usepackage{stmaryrd}

\newcommand{\bs}[1]{\boldsymbol{#1}}
\newcommand{\bb}[1]{\mathbf{#1}}
\newcommand{\pd}[2]{\frac{\partial {#1}}{\partial {#2}}}
\newcommand{\parti}[1]{\pd{}{#1}}
\newcommand{\bigCI}{\mathrel{\text{\scalebox{1.07}{$\perp\mkern-10mu\perp$}}}}
\title{TP6 for Sub-pixel}
\author{Leman FENG\\ Email: flm8620@gmail.com\\Website: lemanfeng.com}

\begin{document}
	\maketitle
	\section*{Ex 15}
	\subsection*{1}
	Notice that $\frac{d x_+^{n+1}}{d x} = (n+1)x_+^{n}, x\neq 0$. So inversely,
	$$
	\int_a^b x_+^n = \frac{1}{n+1} [x^{n+1}_+]_a^b
	$$
	\subsection*{2}
	It's easy to verify this is true for $n=1$.
	Assume it's true for $n$.
	\begin{equation*}
	\begin{split}
	\beta^{n+1}(y) &= \beta^n \star \beta^0 = \frac{1}{n!}\sum_{k=0}^{n+1} \binom{n+1}{k} (-1)^k \int (x-k+\frac{n+1}{2})_+^n \cdot \mathbb{1}_{[-1/2,1/2[}(y-x) dx\\
	&=\frac{1}{n!}\sum_{k=0}^{n+1} \binom{n+1}{k} (-1)^k \int_{y-1/2}^{y+1/2} (x-k+\frac{n+1}{2})_+^n dx\\
	&=\frac{1}{n!}\sum_{k=0}^{n+1} \binom{n+1}{k} (-1)^k \frac{1}{n+1} \int_{y-k+n/2}^{y-k+n/2+1} x_+^{n+1} dx\\
	&=\frac{1}{(n+1)!}\sum_{k=0}^{n+1} \binom{n+1}{k} (-1)^k [x_+^{n+1}]_{y-k+n/2}^{y-k+n/2+1}\\
	&=\frac{1}{(n+1)!}\sum_{k=0}^{n+1} \binom{n+1}{k} (-1)^k ((y-k+n/2+1)_+^{n+1} - (y-k+n/2)_+^{n+1})\\
	&=\frac{1}{(n+1)!}(\sum_{k=0}^{n+1} \binom{n+1}{k} (-1)^k (y-k+(n+2)/2)_+^{n+1} +  \sum_{k=0}^{n+1} \binom{n+1}{k} (-1)^{k+1} (y-(k+1)+(n+2)/2)_+^{n+1})\\
	&=\frac{1}{(n+1)!}(\sum_{k=0}^{n+1} \binom{n+1}{k} (-1)^k (y-k+(n+2)/2)_+^{n+1} +  \sum_{k=1}^{n+2} \binom{n+1}{k-1} (-1)^{k} (y-k+(n+2)/2)_+^{n+1})\\
	\end{split}
	\end{equation*}
	
	We know that
	\begin{equation*}
	\begin{cases}
	\binom{n+2}{k} = \binom{n+1}{k} + \binom{n+1}{k-1} & 1\leq k \leq n+1\\
	\binom{n+2}{k} = \binom{n+1}{k-1} & \leq k = n+2\\
	\binom{n+2}{k} = \binom{n+1}{k} & \leq k = 0
	\end{cases}
	\end{equation*}
	
	So we have
	\begin{equation*}
	\begin{split}
	\beta^{n+1}(y)
	&=\frac{1}{(n+1)!}\sum_{k=0}^{n+2} \binom{n+2}{k} (-1)^k (y-k+(n+2)/2)_+^{n+1}\\
	\end{split}
	\end{equation*}
	so this is also true for $n+1$.
	By recurrence, this is true for all $n\geq 1$
\end{document}
